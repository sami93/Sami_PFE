\chapter*{Introduction générale}
\addcontentsline{toc}{chapter}{Introduction générale} % to include the introduction to the table of content
\markboth{Introduction générale}{} %To redefine the section page head
% concerne la conception, l’analyse, le développement et l’implémentation de méthodes permettant à une machine d’évoluer par un processus systématique. Il consiste à apprendre en tirant des prévisions de fonctionnement ou de comportement à partir de masses de données reçues, en exécutant des algorithmes afin d'identifier des connaissances utiles et de définir d'autres connaissances en utilisant des prédictions avec une très grande précision.\\%
\begin{comment}
L'apprentissage automatique (Machine Learning en anglais) est un champ d'étude de  l’intelligence artificielle, basé sur les domaines de la statistique et de l'informatique.
Il consiste  à développer et à exécuter des algorithmes qui permettent d'inférrer des données inconnues à partir des données connues en utilisant des prédictions.
Partant de ces principes et face aux problèmes rencontrés par Sofrecom dans l'exécution des tâches des prédictions qui vont l'aider à prendre la décision sur la démissions de ses employés dans le futur, un projet d’optimisation du traitement de ces tâches a été engagé.
\vskip Actuellement, le traitement de ces tâches est effectué sur un système d'information qui est caractérisé par une lenteur due notamment à le classement des différentes informations des employés de Sofrecom et au choix du algorithme pour l'extraction des informations utiles et par des difficultés de sauvegarde des prédictions effectuées. L’objectif est de mettre en place un environnement qui permet d’améliorer le traitement des prédictions en temps réel, de faciliter et d’améliorer leur gestion tant au niveau de leur sauvegarde et de leur mise à jour et de réduire les différentes tâches de collecte, d'analyse, d'extraction et d'identification des données et du construction d'algorithme pour effectuer des prédictions.
\vskip Notre rapport s’articulera en quatre chapitres.
Dans le premier chapitre, nous allons présenter le cadre général de notre projet ainsi que la méthodologie adoptée. \\
Le deuxième chapitre sera dédié à l'étude théorique des exigences de notre application de point de vue des besoins fonctionnels et des besoins non fonctionnels.\\
Le troisième chapitre décrira la phase de conception que nous avons illustrée par les diagrammes de séquences et le diagramme de classes. \\
Le quatrième et dernier chapitre réalisation exposera les différentes interfaces de notre application.\\ 
Nous clôturons le présent rapport par une conclusion générale et une présentation de perspectives pour améliorer notre travail effectué.
\end{comment}